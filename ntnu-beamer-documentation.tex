\documentclass[aspectratio=169]{beamer}

\usepackage[english]{babel}
\usepackage[utf8]{inputenc}
\usepackage[T1]{fontenc}
\usepackage{booktabs, listings, pdfpages, biblatex, csquotes}

\lstset{basicstyle=\ttfamily}
\setlength{\parskip}{.5\baselineskip}
\usetheme[slogan=english, style=vertical]{NTNU}
\addbibresource{ref.bib}


\title{\LaTeX{}-Beamer Style for NTNU}
\subtitle{Documentation}
\author{Ronny Bergmann, Torstein Nordgård-Hansen}
\date{\today}

\begin{document}
	\maketitle
	\begin{frame}{Contents}
		\tableofcontents
	\end{frame}


	\section{Introduction}
		\begin{frame}{The \LaTeX{} beamer NTNU theme}
			This \LaTeX beamer theme is based on the Powerpoint templates available at \href{https://innsida.ntnu.no/wiki/-/wiki/English/Create+NTNU+presentations\#section-Create+NTNU+presentations-Powerpoint+templates}{Innsida}.
			The design of these templates is owned by NTNU, and should not be altered
	substantially without checking it with the \href{https://www.ntnu.no/adm/komm}{Communication Division}.

			This documentation assumes that you have your editor and \LaTeX{} compiler set up and running. For \LaTeX{} beamer, the full documentation is available at \url{https://texdoc.org/serve/beamer/0}.

			This documentation:
			\begin{itemize}
				\item Gets you started (slide \#\ref{slide:start})
				\item Illustrates what the three styles look like (slide \#\ref{slide:styles-demo})
				\item Documents all available options and features (slide \#\ref{slide:developer})
			\end{itemize}
		\end{frame}


	\section{Getting Started}
		\begin{frame}[fragile]{Start a presentation}
			\label{slide:start}
			\begin{enumerate}
				\item Copy this folder to the desired location
				\item Configre the language and style in the \lstinline!ntnu-beamer-example.tex! file as indicated by the comments
				\item Comment out or remove unused package imports in the preample
				\item Remove the \lstinline!ntnu-beamer-documentatin.tex! and \lstinline!ntnu-beamer-documentatin.pdf! files, as well as the \lstinline!demos! folder
				\item Rename the \lstinline!ntnu-beamer-example.tex! file as desired
				\item Open the just renamed \lstinline!.tex! file and start writing your presentation, i.e.
				\begin{itemize}
					\item Fill out author and title
					\item Fill or comment out the subtitle
					\item Set the date or leave it at the current date
					\item Start writing your slides, two examples are already given in the template
				\end{itemize}
			\end{enumerate}

			A more elaborate way \emph{instead} of copying all files from step 1 is to
			include them in your \LaTeX\ path.
		\end{frame}


	\section{Example Slides in Different Themes}
	\begin{frame}[fragile]{Different styles}
		\label{slide:styles-demo}
		The following 6 slides illustrate the three styles you can choose,
		first showing the title page and then a usual slide that also explains how to choose this style.

		Note that since \lstinline!slogan=english!, \lstinline!style=plain!, and \lstinline!mathfont=sans! are the defaults, if you use

		\lstinline!\usetheme{NTNU}!

		that is the same as

		\lstinline!\usetheme[slogan=english, style=plain, mathfont=sans]{NTNU}!
	\end{frame}

	{
		\setbeamercolor{background canvas}{bg=}
		\includepdf[pages={1,2}]{demos/demo-ntnu-plain.pdf}
		\includepdf[pages={1,2}]{demos/demo-ntnu-horizontal.pdf}
		\includepdf[pages={1,2}]{demos/demo-ntnu-vertical.pdf}
	}


	\section{Developer Guide}
		\begin{frame}[fragile]{Available options}
			\label{slide:developer}
			You can load the theme using
			
			\lstinline!\usetheme{NTNU}!

			where you have the following options available

			\begin{tabular}{lcl}
				\toprule
				\textbf{Option} & \textbf{Default} &\textbf{Description} \\
				\midrule
				\lstinline!displayframetotal! && the same as \lstinline!frametotal=true! \\
				\lstinline!frametotal=true|false! & \lstinline!false! & toggle display of total number of slides\\
				\lstinline!hideframetotal! && the same as \lstinline!frametotal=false! \\
				\lstinline!slogan=english|norsk! & \lstinline!norsk! & whether to use english or norsk slogan\\
				\lstinline!style= ...! & \lstinline!plain! & styles, see (outer) themes on slide \ref{slide:outer}\\
				\lstinline!mathfont=sans|serif! & \lstinline!sans! & font style in mathmode\\
				\bottomrule
			\end{tabular}
		\end{frame}
		
		\begin{frame}[fragile]{Outer themes}
			\label{slide:outer}
			The communication design team provides four variants, three of which have been implemented
			\\\hfill{\small (outer themes in latex beamer)}

			\begin{description}
				\item[NTNUplain] a very plain style (Default)
				\item[NTNUvertical] with a stripe on the left (current)
				\item[NTNUhorizontal] with a stripe at the bottom
			\end{description}

			All three consist of an outer and an innter theme, these are activated by the style option of the main theme

			\lstinline!style=plain|vertical|horizontal!

			when loading the theme with

			\lstinline!\usetheme{NTNU}!

			as shown on slide \#\ref{slide:styles-demo} and the following example slides.
		\end{frame}
		
		\begin{frame}[fragile]{Special commands}
			\begin{description}
				\item[titlelogo]
				Set the logo on the title page. By default, it is set to the negative english or norwegian logo (depending on the chosen slogan), see the slogan option, e.g.\,
				\lstinline!\titlelogo{ntnu_bredde_eng_neg.png}!
			\end{description}
		\end{frame}
		
		\begin{frame}{Special colors}
	 		These colors are taken from \url{https://innsida.ntnu.no/wiki/-/wiki/Norsk/Farger+i+grafisk+profil}
			\begin{description}
				\item[{\color{NTNUBlue} NTNUBlue}] (\#{\color{NTNUBlue}00509e})
				\item[{\color{NTNULightblue} NTNULightblue}] (\#{\color{NTNULightblue}6096d0})
				\item[{\color{NTNUOrange} NTNUOrange}] (\#{\color{NTNUOrange}ef8114})
				\item[{\color{NTNUPink} NTNUPink}] (\#{\color{NTNUPink}b01b81})
				\item[{\color{NTNUYellow} NTNUYellow}] (\#{\color{NTNUYellow}f7d019})
				\item[{\color{NTNUViolet} NTNUViolet}] (\#{\color{NTNUViolet}482776})
				\item[{\color{NTNUCyan} NTNUCyan}] (\#{\color{NTNUCyan}3cbfbe})
				\item[{\color{NTNUOcher} NTNUOcher}] (\#{\color{NTNUOcher}cfb887})
				\item[{\color{LightGrey} LightGrey}] (\#{\color{LightGrey}bebebe})
			\end{description}
		\end{frame}
		
		\begin{frame}{Further configurations}
			Currently all 3 colors of \lstinline!hyperref! are set to {\color{NTNUBlue} NTNUBlue}, i.\,e.\
			\begin{itemize}
				\item \lstinline!linkcolor! when referring to e.\,g.\ the demo slides slide \#\ref{slide:styles-demo} and following
				\item \lstinline!urlcolor! when using URLs or hyperref links, for example when including this themes \href{https://github.com/ntnu-tex/beamerthementnu}{github page}
				\item \lstinline!citecolor! when citing literature like \footfullcite{TeX-book}.
			\end{itemize}
		\end{frame}
		
		\begin{frame}{Required packages}
			The following packages have to be installed for the theme to work
			\begin{description}
				\item[calc] for a few computational tricks
				\item[ifthen] to define some commands
				\item[pdftexcmds] to define some commands
				\item[opensans] for the font
				\item[tikz] for some graphic tricks
			\end{description}
			This documentation further uses
			\begin{description}
				\item[booktabs] for nice tables
				\item[listings] for code highlighting
				\item[pdfpages] to include the examples
			\end{description}
			This is handleded automatically by \href{https://overleaf.com}{Overleaf} or \href{https://www.tug.org/texlive/}{TeX~Live}
		\end{frame}

\end{document}
